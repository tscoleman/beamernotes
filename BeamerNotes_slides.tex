%% LyX 2.2.3 created this file.  For more info, see http://www.lyx.org/.
%% Do not edit unless you really know what you are doing.
% --- For beamer use the line 1 below, for beamerarticle use the lines 2 below
\documentclass[english]{beamer}
% --- Following four lines for beamerarticle
%\documentclass[english]{article}
%\usepackage{beamerarticle,pgf}
%\usepackage{geometry}
%\geometry{verbose}
% --- End of additional lines for beamerarticle
% IMPORTANT 
%    - errors when switching from "beamer" to "beamerarticle"
%        - auxiliary files left after compiling may produce errors - TRASH ALL AUXILIARY FILES between compiles
%        - in TexShop got to <File> <Trash Aux Files>
%    - You may need to compile twice to get Table of Contents properly (at least in TexShop one needs to)
% Additional comments:
% Set \mode* at beginning of the document - ignores non-Frame text in beamer mode
% Use ``shadowbox'' to highlight & demarcate slides

\usepackage[T1]{fontenc}
\usepackage[latin9]{inputenc}
\setcounter{secnumdepth}{3}
\setcounter{tocdepth}{3}
\usepackage{fancybox}
\usepackage{calc}

\makeatletter
%%%%%%%%%%%%%%%%%%%%%%%%%%%%%% Textclass specific LaTeX commands.
 % this default might be overridden by plain title style
 \newcommand\makebeamertitle{\frame{\maketitle}}%
 % (ERT) argument for the TOC
 \AtBeginDocument{%
   \let\origtableofcontents=\tableofcontents
   \def\tableofcontents{\@ifnextchar[{\origtableofcontents}{\gobbletableofcontents}}
   \def\gobbletableofcontents#1{\origtableofcontents}
 }

%%%%%%%%%%%%%%%%%%%%%%%%%%%%%% User specified LaTeX commands.
\usetheme{CambridgeUS}
\usecolortheme{seagull}
\useinnertheme{circles}

\makeatother

\usepackage{babel}
\begin{document}

\title[\emph{Beamer Example}]{\emph{Beamer Notes + Slides \textendash{} Example}}

\author[Coleman]{Thomas S. Coleman}

\institute[\emph{Harris}]{\emph{Harris School of Public Policy}}

\date{October 12, 2017}

\makebeamertitle
\mode*

\noindent\shadowbox{\begin{minipage}[t]{1\columnwidth - 2\fboxsep - 2\fboxrule - \shadowsize}%
\begin{frame}{Creating Notes \& Slides}

Goal for Lectures
\begin{itemize}
\item Slides for Lectures
\item Notes (``script'') for myself
\item Ability to print (and then post) either \emph{Slides} or \emph{Notes} 
\end{itemize}
All in the same file
\begin{itemize}
\item Ability to easily print either Slides or Notes+Slides
\item For notes, include both slides and my notes
\item For notes, slides must display as they look in presentation mode (but
smaller), and also highlighted \& demarcated
\end{itemize}
Different from Beamer ``Handout'' 

\end{frame}
%
\end{minipage}}

For many years I have yearned for a way to create a single document
that has my full slides (with animations or any other bells-and-whistles)
together with the notes and my ``script'' for the presentation \textendash{}
my notes to myself for what I want to say. 
\begin{itemize}
\item Slides using Beamer \textendash{} full power \& functionality of Beamer
\item Text and notes and anything else I need for myself as reminders, script,
and background for the presentation
\end{itemize}
I have a few objectives for this combination of notes + slides:
\begin{itemize}
\item Slides should harness the full power and functionality of Beamer
\item Notes should have the full functionality of Latex \textendash{} equations,
graphics, ... 
\item Printed version of the notes should show the slides as they appear
in the presentation, highlighted and demarcated from the notes
\item Printed version of the notes should have slides compact and compressed
relative to the slides so that everything can be printed with limited
damage to our forests. 
\end{itemize}
\noindent\shadowbox{\begin{minipage}[t]{1\columnwidth - 2\fboxsep - 2\fboxrule - \shadowsize}%
\begin{frame}{Solution \textendash{} \emph{beamerarticle} }

Single File
\begin{itemize}
\item Contains both the Beamer Slides \& text, graphics, equations, ...
\item Trick is using \emph{beamerarticle} \& \emph{\textbackslash{}mode{*}} 
\end{itemize}
Set \emph{\textbackslash{}mode{*}} at beginning of the document \textendash{}
ignores text in \emph{beamer} mode

Use ``shadowbox'' to highlight \& demarcate slides

For printing Slides, 
\begin{itemize}
\item document class \textquotedblleft \emph{beamer}\textquotedblright{}
\item remove the \textquotedbl{}\emph{\textbackslash{}usepackage\{beamerarticle,pgf\}}\textquotedbl{}
\end{itemize}
For printing Notes+Slides
\begin{itemize}
\item document class \textquotedblleft \emph{article}\textquotedblright{}
\item Insert ``\emph{\textbackslash{}usepackage\{beamerarticle,pgf\}}'' 
\end{itemize}
\end{frame}
%
%
\end{minipage}}
\begin{enumerate}
\item \emph{\textbackslash{}mode{*}} 
\begin{enumerate}
\item Set this command at the beginning of your document
\item For \emph{article} class this does nothing, while for \emph{beamer}
class (presentations) it ignores all text outside frames. 
\item Section 21.3 of Beamer User Guide: ``The effect of this mode is to
ignore all text outside frames in the presentation modes. In article
mode it has no effect.'' 
\end{enumerate}
\item Minibox to highlight \& demarcate slides
\begin{enumerate}
\item I want to clearly demarcate slides \textendash{} use ``shadowbox''
minipage
\begin{quotation}
\textbackslash{}noindent\textbackslash{}shadowbox\{\textbackslash{}begin\{minipage\}{[}t{]}\{1\textbackslash{}columnwidth
- 2\textbackslash{}fboxsep - 2\textbackslash{}fboxrule - \textbackslash{}shadowsize\}\%

\textbackslash{}begin\{frame\}\{Title\}

My slide content here

\textbackslash{}end\{frame\}

\%

\textbackslash{}end\{minipage\}\}
\end{quotation}
\end{enumerate}
\item Print document to \emph{pdf} using either \emph{beamer} class (for
slides) or \emph{article} class + \emph{beamerarticle} package for
slides+notes
\begin{enumerate}
\item Slides: \emph{beamer} class
\begin{quotation}
\textbf{\textbackslash{}documentclass{[}english{]}\{beamer\}}

\textbackslash{}usepackage{[}T1{]}\{fontenc\}

\textbackslash{}usepackage{[}latin9{]}\{inputenc\}

\textbackslash{}setcounter\{secnumdepth\}\{3\}

\textbackslash{}setcounter\{tocdepth\}\{3\}

\textbackslash{}usepackage\{fancybox\}

\textbackslash{}usepackage\{calc\}

\textbackslash{}makeatletter

\%\%\% Textclass specific LaTeX commands.
\end{quotation}
\item Notes+Slides: \emph{article} class + \emph{beamerarticle} package
\begin{quotation}
\textbf{\textbackslash{}documentclass{[}english{]}\{article\}}

\textbackslash{}usepackage{[}T1{]}\{fontenc\}

\textbackslash{}usepackage{[}latin9{]}\{inputenc\}

\textbf{\textbackslash{}usepackage\{geometry\}}

\textbf{\textbackslash{}geometry\{verbose\}}

\textbackslash{}setcounter\{secnumdepth\}\{3\}

\textbackslash{}setcounter\{tocdepth\}\{3\}

\textbackslash{}usepackage\{fancybox\}

\textbackslash{}usepackage\{calc\}

\textbackslash{}makeatletter

\%\%\% Textclass specific LaTeX commands.

\textbf{\textbackslash{}usepackage\{beamerarticle,pgf\}}
\end{quotation}
\item See section 21.2 of the Beamer User Guide. 
\end{enumerate}
\item I use \emph{Lyx} and here it is particularly easy
\begin{enumerate}
\item Main file, with all my slides and text, title ``MySlides\_slides.lyx''
\begin{enumerate}
\item Set Document Class \emph{beamer} 
\end{enumerate}
\item Notes file: simple ``cover'' file that ``includes'' my notes file,
title ``MySlides\_notes.lyx''
\begin{enumerate}
\item ``Include: MySlides\_slides.lyx''
\item Set Document Class \emph{beamerarticle} 
\end{enumerate}
\item When I print from ``\_slides.lyx'' I get slides. When I print from
``\_notes.lyx'' I get Notes+Slides. Easy
\begin{enumerate}
\item NB - youi may get a warning when printing from ``\_notes.lyx'' about
a conflict in classes (beamer versus article) but that is OK \textendash{}
you are essentially overwriting the ``beamer'' class with ``article''
class and that is what you want to do. 
\end{enumerate}
\end{enumerate}
\end{enumerate}

\end{document}
